\begin{conclusions}
La presente tesis ha confirmado la tesis de que, aunque el algoritmo \textit{Conflict-Driven Clause Learning} (CDCL) representa un avance significativo en la resolución de problemas SAT al combinar aprendizaje de cláusulas y selección heurística de variables, persiste un dilema fundamental en la dependencia de la heurística inicial para la selección de variables. Este aspecto condiciona el rendimiento del \textit{solver}, especialmente en problemas con estructuras complejas y gran número de variables, donde la probabilidad de TIMEOUT y los tiempos de resolución se incrementan notablemente, independientemente de la heurística aplicada.

Los resultados estadísticos obtenidos a través de pruebas no paramétricas, regresiones logísticas, análisis de varianza y modelos lineales confirman que el número de variables es el factor más determinante en la eficiencia del \textit{solver}, seguido por la densidad de cláusulas, cuyo efecto es moderado y en algunos casos inverso, y el tamaño promedio de cláusula, cuya influencia varía según interacciones con otras características. En cuanto al desempeño de las heurísticas, no se identificó una estrategia universalmente óptima; sin embargo, la heurística DLIS sin reinicios mostró un mejor rendimiento promedio y menor tasa de TIMEOUT en problemas medianos y grandes con ciertas estructuras, mientras que las variantes basadas en VSIDS presentaron mayor variabilidad y peor desempeño en instancias densas y grandes, aunque con eficacia en problemas pequeños y medianos.

En conclusión, la investigación subraya la importancia crítica de adaptar la selección heurística y las estrategias de optimización a las características estructurales del problema para maximizar la eficiencia del \textit{solver} CDCL. Los hallazgos estadísticos robustos, reafirman la orientaci\'on hacia el desarrollo de heurísticas adaptativas y modelos predictivos que consideren el tamaño y la estructura del problema, abriendo camino a solucionadores más inteligentes y generalizables. Este trabajo aporta una base sólida para futuras investigaciones que busquen superar el cuello de botella en la selección de variables y perfeccionar la sinergia entre aprendizaje y heurística en algoritmos de satisfacibilidad booleana.

\end{conclusions}
