\begin{conclusions}
El dilema en CDCL
CDCL mitiga parcialmente este problema mediante el aprendizaje de cláusulas, pero no elimina la dependencia de la selección inicial de variables. Por ejemplo:

Si VSIDS elige variables periféricas en un problema con núcleos críticos (ej: PHP), el solver gastará recursos en regiones irrelevantes.

Si DLIS prioriza literales frecuentes en problemas con restricciones jerárquicas (ej: scheduling), perderá la capacidad de explotar correlaciones locales.

Esta interdependencia entre heurísticas y estructura del problema explica por qué, a pesar de los avances en CDCL, no existe una estrategia universalmente óptima. La selección de variables sigue siendo un cuello de botella teórico y práctico, especialmente al escalar a miles de variables con relaciones complejas.

La introducción de CDCL marcó un avance al reemplazar el retroceso (backtrack) cronológico con uno dirigido por conflictos, pero su éxito está ligado a la sinergia entre aprendizaje y selección de variables. Mientras las cláusulas aprendidas reducen el espacio de búsqueda, las heurísticas de selección determinan cómo se navega en él. Un desbalance entre estos componentes condena al solver a un rendimiento subóptimo, perpetuando la necesidad de estudios comparativos como el propuesto en esta tesis.

La efectividad de los solucionadores modernos de satisfacibilidad booleana (SAT) descansa en gran medida en la capacidad de integrar de manera sin\'ergica estrategias de optimizaci\'on y heur\'isticas dentro del esquema general del algoritmo Conflict-Driven Clause Learning (CDCL). Lejos de constituir componentes aislados, estas mejoras operan como m\'odulos interdependientes que refinan y potencian el comportamiento global del algoritmo, permiti\'endole escalar de manera eficiente frente a instancias de gran complejidad.

Estrategias como los reinicios adaptativos, ejemplificados por los esquemas de Luby o Glucose-style, permiten una regulaci\'on din\'amica del balance entre exploraci\'on y explotaci\'on dentro del espacio de b\'usqueda. Al reconocer patrones de estancamiento mediante m\'etricas como el LBD, estos mecanismos inducen reinicios calculados que promueven la reorientaci\'on del proceso de deducci\'on, sin perder el conocimiento acumulado en forma de cl\'ausulas aprendidas.

Simult\'aneamente, estructuras como Two Watched Literals (TWL) reformulan la propagaci\'on unitaria, eliminando redundancias computacionales y concentrando la atenci'on en un subconjunto relevante de cl\'ausulas. Esto reduce de forma considerable el costo por iteraci\'on, acelerando la inferencia sin comprometer la correcci\'on de las deducciones.

Estas estrategias, cuando se integran con coherencia en la arquitectura del solucionador, transforman el comportamiento de CDCL en una maquinaria altamente adaptativa. Cada componente contribuye a un objetivo com\'un: reducir el tiempo necesario para alcanzar una soluci\'on, ya sea satisfactible o insatisfactible, mediante una navegaci\'on informada y eficiente del espacio de b\'usqueda. Esta modularidad permite, adem\'as, la extensibilidad del algoritmo, favoreciendo la experimentaci\'on y la incorporaci\'on de nuevas heur\'isticas en funci\'on de las demandas de cada dominio de aplicaci\'on.

En suma, la integraci\'on de heur\'isticas avanzadas y estrategias de optimizaci\'on en el marco de CDCL no s\'olo representa una mejora en rendimiento, sino que constituye una evoluci\'on conceptual en el dise\~no de algoritmos de b\'usqueda, marcando un camino claro hacia solucionadores m\'as inteligentes, robustos y generalizables.


\end{conclusions}
