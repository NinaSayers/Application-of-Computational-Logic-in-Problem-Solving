\begin{resumen}
Esta tesis realiza un an\'alisis comparativo de las heur\'isticas de decisi\'on de variables VSIDS (Variable State Independent Decaying Sum) y DLIS (Dynamic Largest Individual Sum) en \textit{solvers} SAT basados en \textit{Conflict-Driven Clause Learning} (CDCL). El estudio examina ambas heur\'isticas con y sin estrategias de \textit{restart}, utilizando como plataforma de evaluaci\'on el \textit{solver} CaDiCaL. 

Se implement\'o DLIS en CaDiCaL mediante modificaciones que permitieron su integraci\'on con la estructura existente del \textit{solver} y sus otras heur\'isticas. Para la evaluaci\'on, se desarroll\'o un \textit{benchmark} de 135 problemas SAT mediante un \textit{script} Python, abarcando diversas categor\'ias: \textit{random }k-sat, \textit{random biased} k-sat, \textit{pingeonhole}, \textit{graph coloring}, \textit{parity} XOR y BCM flip flop. Estos problemas presentan variaciones en sus estructuras, incluyendo n\'umeros de variables (desde decenas hasta miles), cantidad de cl\'ausulas y densidad de restricciones.

Cada combinaci\'on heur\'istica (VSIDS/DLIS, con/sin \textit{restart}) se ejecut\'o sobre todos los problemas mediante \textit{flags} espec\'ificos de CaDiCaL, estableciendo un \textit{timeout} de 2 minutos por ejecuci\'on. Los resultados se almacenaron sistem\'aticamente en un CSV, registrando: nombre del problema, heur\'istica empleada, resultado (SATISFIABLE/UNSATISFIABLE/TIMEOUT), tiempo de resoluci\'on, cantidad de variables, cantidad de cl\'ausulas, densidad, entre otros.

El an\'alisis estad\'istico revel\'o que el rendimiento depende m\'as de la estructura intr\'inseca del problema que de la heur\'istica espec\'ifica o el uso de \textit{restart}. No se observaron diferencias significativas entre DLIS con/sin \textit{restart}, y VSIDS con/sin \textit{restart}. Sin embargo, VSIDS exhibi\'o mayor varianza en los tiempos de ejecuci\'on, mientras DLIS demostr\'o ser m\'as eficiente para problemas espec\'ificos como los de coloraci\'on de grafos y estructuras \textit{XOR-parity}. Estos hallazgos sugieren que la selecci\'on \'optima de heur\'isticas deber\'ia considerar las caracter\'isticas estructurales de los problemas SAT a resolver.
\end{resumen}

\begin{abstract}
This Bachelor's Thesis conducts a comparative analysis of variable decision heuristics VSIDS (Variable State Independent Decaying Sum) and DLIS (Dynamic Largest Individual Sum) in Conflict-Driven Clause Learning (CDCL) SAT solvers. The study examines both heuristics with and without restart strategies, using the CaDiCaL solver as the evaluation platform.

DLIS was implemented in CaDiCaL through modifications that enabled its integration with the solver's existing structure and other heuristics. For evaluation, a benchmark of 135 SAT problems was developed using a Python script, covering diverse categories: random k-sat, random biased k-sat, pigeonhole principle, graph coloring, parity XOR, and BCM flip flop. These problems feature structural variations including variable counts (from tens to thousands), clause quantities, and constraint densities.

Each heuristic combination (VSIDS/DLIS, with/without restart) was executed on all problems using specific CaDiCaL flags, with a 2-minute timeout per execution. Results were systematically stored in a CSV file, recording: problem name, heuristic used, outcome (SATISFIABLE/UNSATISFIABLE/TIMEOUT), resolution time, variable count, clause count, density and others.

Statistical analysis revealed that performance depends more on the intrinsic structure of the problem than on the specific heuristic or restart usage. No significant differences were observed between DLIS with/without restart, and VSIDS with/without restart. However, VSIDS exhibited greater variance in execution times, while DLIS proved more efficient for specific problems such as graph coloring and XOR-parity structures. These findings suggest that optimal heuristic selection should consider the structural characteristics of target SAT problems.
\end{abstract}