\begin{opinion}
Contin\'ua la investigaci\'on en Lógica Computacional de métodos cada vez más eficientes de solución de problemas con satisfacción de restricciones, que incluyen problemas muy importantes de optimización combinatoria. La búsqueda de solucionadores más eficientes  basados en SAT constituyen un área relevante de investigación de la Inteligencia Artificial, dada la decisiva aplicación de tales métodos a la solución de problemas reales complejos de planificación, horarios, asignación, logísticos, etc, que requieren la optimización de recursos y costos.

Un tema muy importante, aún no tratado por nuestro grupo, es la Introducción de heurísticas para la selección de variables de decisión en solucionadores CDCL-SAT. El trabajo de diploma que presenta para su defensa la estudiante Massiel Paz Otaño lleva a cabo tal indagación, obteniendo resultados en el análisis comparativo de ciertas estrategias que se han diseñado al respecto.

Es de se\~nalar la motivación y aplicación con los que la estudiante enfrentó la tarea sobre un tema complejo y como fue satisfaciendo todos los requisitos que le fueron planteados para su realización. Considero que la estudiante Massiel Paz Otaño ha alcanzado el nivel de profesionalidad que exige obtener el título de Lic. en Ciencia de la Computación y por tal motivo solicitamos la calificación de Excelente (5) para su trabajo de diploma.
La Habana, 15 de junio de 2025

\begin{figure}[ht]
    \includegraphics[width=0.3\textwidth]{Graphics/picture1.png}
\end{figure}

Dr. Luciano García Garrido\\
Profesor Titular Consultante\\
Facultad de Matemática y Computación\\
Universidad de La Habana, Cuba

\end{opinion}

