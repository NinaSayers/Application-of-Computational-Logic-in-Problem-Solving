\chapter{Algoritmos}
\label{chapter:algorithms}

\section{DP/DPLL}




\section{CDCL}

%\label{code:cdcl-sat-solver}
\begin{lstlisting}
class SATSolver:
    def __init__(self, formula):
        """
        Initializes the SAT solver.
        
        Parameters:
          formula: A list of clauses, where each clause is represented as a list of integers.
                   A positive integer i represents the variable x_i, and a negative integer -i represents -x_i.
        """
        # Copy the formula so that learned clauses can be appended.
        self.formula = formula[:]  
        self.assignments = {}    # Maps variable -> True/False assignment.
        self.levels = {}         # Maps variable -> decision level at which it was assigned.
        self.reasons = {}        # Maps variable -> clause that forced the assignment (None for decision variables).
        self.decision_level = 0  # Current decision level.
        self.decision_stack = [] # Stack storing tuples (variable, assigned value, decision level).

    def literal_value(self, literal):
        """
        Evaluates a literal given the current partial assignment.
        
        Returns:
          True if the literal is assigned True,
          False if the literal is assigned False,
          None if the variable is unassigned.
        """
        var = abs(literal)
        if var not in self.assignments:
            return None
        # For a positive literal, the assignment is the value; for a negative literal, invert the assignment.
        return self.assignments[var] if literal > 0 else not self.assignments[var]

    def check_clause(self, clause):
        """
        Determines the status of a clause with respect to current assignments.
        
        Returns a tuple (status, literal) where status is one of:
          - `satisfied': Clause is already True under the assignment.
          - `conflict': All literals are assigned False (the clause is unsatisfied).
          - `unit': Exactly one literal is unassigned while all others are False (this literal must be True).
          - `undefined': The clause is neither satisfied, conflicting, nor unit.
        """
        satisfied = False
        unassigned_count = 0
        unit_literal = None
        for literal in clause:
            val = self.literal_value(literal)
            if val is True:
                return (`satisfied', None)
            if val is None:
                unassigned_count += 1
                unit_literal = literal  # Last seen unassigned literal.
        if unassigned_count == 0:
            return (`conflict', None)
        if unassigned_count == 1:
            return (`unit', unit_literal)
        return (`undefined', None)

    def unit_propagate(self):
        """
        Repeatedly applies unit propagation.
        
        Returns:
          A conflicting clause if a conflict is found during propagation; otherwise, returns None.
        """
        changed = True
        while changed:
            changed = False
            for clause in self.formula:
                status, unit_literal = self.check_clause(clause)
                if status == 'conflict':
                    # A clause is unsatisfied => conflict!
                    return clause
                elif status == `unit':
                    var = abs(unit_literal)
                    if var not in self.assignments:
                        # Determine the value needed to satisfy the unit clause.
                        value = (unit_literal > 0)
                        self.assignments[var] = value
                        self.levels[var] = self.decision_level
                        self.reasons[var] = clause  # Store the clause as the reason for this assignment.
                        self.decision_stack.append((var, value, self.decision_level))
                        changed = True
        return None

    def pick_branching_variable(self):
        """
        Selects the next unassigned variable found in the formula.
        
        (In a production solver, better heuristics like VSIDS are used.)
        """
        variables = set()
        for clause in self.formula:
            for literal in clause:
                variables.add(abs(literal))
        for var in variables:
            if var not in self.assignments:
                return var
        return None

    def resolve(self, clause1, clause2, pivot):
        """
        Performs the resolution on two clauses over the pivot literal.
        
        Specifically, it returns:
          (clause1 \ {pivot}) U (clause2 \ {-pivot})
        """
        new_clause = []
        for lit in clause1:
            if lit == pivot:
                continue
            if lit not in new_clause:
                new_clause.append(lit)
        for lit in clause2:
            if lit == -pivot:
                continue
            if lit not in new_clause:
                new_clause.append(lit)
        return new_clause

    def conflict_analysis(self, conflict_clause): 
        """
        Conducts conflict analysis to find the First UIP.
        """
        learned_clause = conflict_clause.copy()
        current_level = self.decision_level

        while True:
            # Collect literals in the learned clause assigned at the current level
            current_level_lits = [
                lit for lit in learned_clause 
                if self.levels.get(abs(lit), -1) == current_level
            ]
            if len(current_level_lits) <= 1:
                break

            # Find the most recently assigned literal in current_level_lits
            last_literal = None
            # Iterate through assignments in reverse order (most recent first)
            for var, _, lvl in reversed(self.decision_stack):
                if lvl != current_level:
                    continue
                # Check if this variable is in current_level_lits
                for lit in current_level_lits:
                    if abs(lit) == var:
                        last_literal = lit
                        break
                if last_literal is not None:
                    break

            if last_literal is None:
                break  # No resolvable literals (should not happen)

            # Resolve with the reason clause of last_literal
            reason_clause = self.reasons.get(abs(last_literal))
            if reason_clause is None:
                break  # Decision literal; cannot resolve further

            learned_clause = self.resolve(learned_clause, reason_clause, last_literal)

        # Determine the backjump level
        backjump_level = 0
        for lit in learned_clause:
            lvl = self.levels.get(abs(lit), 0)
            if lvl != current_level and lvl > backjump_level:
                backjump_level = lvl

        return learned_clause, backjump_level

    def backjump(self, level):
        """
        Backtracks the search to the given decision level by undoing assignments above that level.
        """
        new_stack = []
        for var, value, lvl in self.decision_stack:
            if lvl > level:
                if var in self.assignments:
                    del self.assignments[var]
                if var in self.levels:
                    del self.levels[var]
                if var in self.reasons:
                    del self.reasons[var]
            else:
                new_stack.append((var, value, lvl))
        self.decision_stack = new_stack

    def solve(self):
        """
        The main solving loop which alternates between unit propagation, conflict analysis, and branching.
        
        Returns:
          A satisfying assignment as a dictionary mapping variables to Boolean values if the formula is SAT;
          Otherwise, returns None indicating the formula is UNSAT.
        """
        while True:
            conflict = self.unit_propagate()
            if conflict:
                if self.decision_level == 0:
                    # Conflict at level 0 indicates an unsolvable (UNSAT) condition.
                    return None
                learned_clause, backjump_level = self.conflict_analysis(conflict)
                # Learn the clause by adding it to the formula.
                self.formula.append(learned_clause)
                # Backjump to the appropriate decision level.
                self.backjump(backjump_level)
                self.decision_level = backjump_level
            else:
                var = self.pick_branching_variable()
                if var is None:
                    return self.assignments
                self.decision_level += 1
                # For this example, we simply decide that the variable is True.
                self.assignments[var] = True
                self.levels[var] = self.decision_level
                self.reasons[var] = None  # Decision assignments have no reason clause.
                self.decision_stack.append((var, True, self.decision_level))
\end{lstlisting}


%\label{code:cdcl-dlis-sat-solver}
\begin{lstlisting}
def pick_branching_variable(self):
    """
    Selects the next unassigned variable using the DLIS heuristic.
    Returns (variable, value) to assign, or None if all variables are assigned.
    """
    pos_counts = {}
    neg_counts = {}
    # Count occurrences in unsatisfied clauses
    for clause in self.formula:
        status, _ = self.check_clause(clause)
        if status == 'satisfied':
            continue
        for lit in clause:
            var = abs(lit)
            if var not in self.assignments:
                if lit > 0:
                    pos_counts[var] = pos_counts.get(var, 0) + 1
                else:
                    neg_counts[var] = neg_counts.get(var, 0) + 1
    # Collect all variables in the formula to find unassigned ones not in any clause
    all_vars = set()
    for clause in self.formula:
        for lit in clause:
            all_vars.add(abs(lit))
    unassigned_vars = [var for var in all_vars if var not in self.assignments]
    if not unassigned_vars:
        return None
    # For variables not in pos/neg counts, set counts to 0
    for var in unassigned_vars:
        if var not in pos_counts:
            pos_counts[var] = 0
        if var not in neg_counts:
            neg_counts[var] = 0
    # Score variables based on DLIS heuristic
    scores = []
    for var in unassigned_vars:
        pos = pos_counts[var]
        neg = neg_counts[var]
        max_count = max(pos, neg)
        total = pos + neg
        scores.append((-max_count, -total, var))  # Negative for ascending sort
    scores.sort()  # Sorts by max_count (desc), then total (desc), then var (asc)
    var = scores[0][2]
    value = pos_counts[var] > neg_counts[var]
    return (var, value)

\end{lstlisting}


%\label{code:cdcl-vsids-sat-solver}
\begin{lstlisting}
def pick_branching_variable(self):
    """
    Selects the next unassigned variable using the VSIDS heuristic (highest activity).
    """
    candidates = []
    for var in self.activity:
        if var not in self.assignments:
            candidates.append(var)
    if not candidates:
        return None
    # Select the candidate with the highest activity; in case of tie, choose the smallest variable.
    max_activity = max(self.activity[var] for var in candidates)
    best_vars = [var for var in candidates if self.activity[var] == max_activity]
    best_vars.sort()  # Deterministic tie-breaking by choosing the smallest variable
    return best_vars[0]
\end{lstlisting}


%\label{code:init-struct-for-vsids}
\begin{lstlisting}
# previus code
    for clause in self.formula:
        for lit in clause:
            var = abs(lit)
            if var not in self.activity:
                self.activity[var] = 0.0
\end{lstlisting}

%\label{code:actual-score-vsids}
\begin{lstlisting}
decay_factor = 0.95
while True:
    conflict = self.unit_propagate()
    if conflict:
        if self.decision_level == 0:
            # Conflict at level 0 indicates an unsolvable (UNSAT) condition.
            return None
        learned_clause, backjump_level = self.conflict_analysis(conflict)
        # Learn the clause by adding it to the formula.
        self.formula.append(learned_clause)
        # Update activities for variables in the learned clause
        for lit in learned_clause:
            var = abs(lit)
            self.activity[var] += 1.0
        # Decay all activities
        for var in self.activity:
            self.activity[var] *= decay_factor
        # Backjump to the appropriate decision level.
        self.backjump(backjump_level)
        self.decision_level = backjump_level
# rest of code
\end{lstlisting}


%\label{code:cdcl-luby-sat-solver}
\begin{lstlisting}
# restart_luby.py

from collections import deque

def luby(u, k):
    """
    Generates the k-th value of the Luby sequence multiplied by u (unit run).
    """
    def _luby(i):
        # Encuentra el mayor j tal que i = 2^j - 1
        j = 1
        while (1 << j) - 1 < i:
            j += 1
        if i == (1 << j) - 1:
            return 1 << (j - 1)
        return _luby(i - (1 << (j - 1)) + 1)
    return u * _luby(k)

class SATSolverLuby:
    def __init__(self, formula, unit_run=100):
        from restart_luby import luby  # si ejecutas desde fuera
        self.formula = formula[:]  
        self.assignments = {}
        self.levels = {}
        self.reasons = {}
        self.decision_level = 0
        self.decision_stack = []
        # Luby restart parameters
        self.unit_run = unit_run
        self.luby_idx = 1
        self.conflicts_since_restart = 0
        self.next_restart = luby(self.unit_run, self.luby_idx)

    # the same functions (literal_value, check_clause, unit_propagate,
    # pick_branching_variable, resolve, conflict_analysis, backjump)

    def solve(self):
        while True:
            conflict = self.unit_propagate()
            if conflict:
                self.conflicts_since_restart += 1
                if self.decision_level == 0:
                    return None
                learned_clause, backjump_level = self.conflict_analysis(conflict)
                self.formula.append(learned_clause)
                self.backjump(backjump_level)
                self.decision_level = backjump_level

                # restart?
                if self.conflicts_since_restart >= self.next_restart:
                    # Restart: clear assignments, preserve learned clauses
                    self.assignments.clear()
                    self.levels.clear()
                    self.reasons.clear()
                    self.decision_stack.clear()
                    self.decision_level = 0
                    # Prepare next umbral
                    self.luby_idx += 1
                    self.next_restart = luby(self.unit_run, self.luby_idx)
                    self.conflicts_since_restart = 0
            else:
                var = self.pick_branching_variable()
                if var is None:
                    return self.assignments
                self.decision_level += 1
                self.assignments[var] = True
                self.levels[var] = self.decision_level
                self.reasons[var] = None
                self.decision_stack.append((var, True, self.decision_level))
\end{lstlisting}



De igual modo, una implementación para la estrategia \textit{Glucose-Style (LBD-based)} sería como la que se muestra a continuación:

\begin{lstlisting}
# restart_glucose.py

class SATSolverGlucose:
    def __init__(self, formula, lbd_window=50):
        self.formula = formula[:]  
        self.assignments = {}
        self.levels = {}
        self.reasons = {}
        self.decision_level = 0
        self.decision_stack = []
        # Glucose-style parameters
        self.lbd_history = []
        self.window_size = lbd_window
        self.prev_avg_lbd = float('inf')

    #same base functions: literal_value, check_clause, unit_propagate, pick_branching_variable, resolve, backjump

    def conflict_analysis(self, conflict_clause):
        learned_clause, backjump_level = super().conflict_analysis(conflict_clause)
        # Calcular LBD (Literal Block Distance)
        levels = { self.levels.get(abs(l), 0) for l in learned_clause }
        lbd = len(levels)
        # Mantener ventana de LBDs
        self.lbd_history.append(lbd)
        if len(self.lbd_history) > self.window_size:
            self.lbd_history.pop(0)
        return learned_clause, backjump_level

    def should_restart(self):
        if len(self.lbd_history) < self.window_size:
            return False
        curr_avg = sum(self.lbd_history) / len(self.lbd_history)
        # Reiniciar si la media de LBD sube respecto al ciclo anterior
        if curr_avg > self.prev_avg_lbd:
            self.prev_avg_lbd = curr_avg
            return True
        self.prev_avg_lbd = curr_avg
        return False

    def solve(self):
        while True:
            conflict = self.unit_propagate()
            if conflict:
                if self.decision_level == 0:
                    return None
                learned_clause, backjump_level = self.conflict_analysis(conflict)
                self.formula.append(learned_clause)
                self.backjump(backjump_level)
                self.decision_level = backjump_level

                # Glucose-style restart
                if self.should_restart():
                    self.assignments.clear()
                    self.levels.clear()
                    self.reasons.clear()
                    self.decision_stack.clear()
                    self.decision_level = 0

            else:
                var = self.pick_branching_variable()
                if var is None:
                    return self.assignments
                self.decision_level += 1
                self.assignments[var] = True
                self.levels[var] = self.decision_level
                self.reasons[var] = None
                self.decision_stack.append((var, True, self.decision_level))

\end{lstlisting}



%\label{code:cdcl-twl-sat-solver}
\begin{lstlisting}
import formulas as f
from collections import defaultdict, deque

class SATSolver:
    def __init__(self, formula):
        """
        Initializes the SAT solver.

        Parameters:
          formula: A list of clauses, where each clause is represented as a list of integers.
                   A positive integer i represents the variable x_i, and a negative integer -i represents -x_i.
        """
        # Copy the formula so that learned clauses can be appended.
        self.clauses = [list(c) for c in formula]
        self.assignments = {}    # Maps variable -> True/False assignment.
        self.levels = {}         # Maps variable -> decision level at which it was assigned.
        self.reasons = {}        # Maps variable -> clause that forced the assignment (None for decision vars).
        self.decision_level = 0  # Current decision level.
        self.decision_stack = [] # Stack of (variable, value, level).

        # Two-Watched Literals: map literal -> list of clause indices watching it
        self.watches = defaultdict(list)
        self._init_watches()

    def _init_watches(self):
        """Initialize two watched literals per clause."""
        for ci, clause in enumerate(self.clauses):
            # If clause has only one literal, watch it twice.
            w0 = clause[0]
            w1 = clause[1] if len(clause) > 1 else clause[0]
            self.watches[w0].append(ci)
            self.watches[w1].append(ci)

    def literal_value(self, literal):
        """
        Evaluate a literal under current partial assignment.
        Returns True, False, or None if unassigned.
        """
        var = abs(literal)
        if var not in self.assignments:
            return None
        return self.assignments[var] if literal > 0 else not self.assignments[var]

    def check_clause(self, clause):
        """
        Determine clause status: 'satisfied', 'conflict', 'unit', or 'undefined'.
        If 'unit', also return the unit literal.
        """
        unassigned = 0
        last = None
        for lit in clause:
            val = self.literal_value(lit)
            if val is True:
                return ('satisfied', None)
            if val is None:
                unassigned += 1
                last = lit
        if unassigned == 0:
            return ('conflict', None)
        if unassigned == 1:
            return ('unit', last)
        return ('undefined', None)

    def _enqueue(self, var, value, level, reason):
        """
        Assign var=value at given level with reason and push onto decision stack.
        Returns the corresponding literal for propagation.
        """
        self.assignments[var] = value
        self.levels[var] = level%
        self.reasons[var] = reason%
        self.decision_stack.append((var, va%lue, level))
        return var if value else -var%

    def unit_propagate(self):%
        """%
        Perform unit propagation using two-%watched literals.
        Returns a conflicting clause if con%flict, else None.
        """%
        queue = deque()%
        # Enqueue all literals assigned at %current level
        for var, val, lvl in self.decision_%stack:
            if lvl == self.decision_level:%
                queue.append(var if val els%e -var)

        while queue:%
            lit = queue.popleft()%
            lit_false = -lit%
            # We iterate over a snapshot si%nce watch list may change
            watchers = list(self.watches[li%t_false])
            for ci in watchers:%
                clause = self.clauses[ci]%
                # Try to find a new literal% to watch instead of lit_false
                found_replacement = False%
                for l in clause:%
                    if l == lit_false:%
                        continue%
                    if self.literal_value(l%) is not False:
                        # relocate watch fr%om lit_false to l
                        self.watches[l].app%end(ci)
                        self.watches[lit_fa%lse].remove(ci)
                        found_replacement =% True
                        break%
                if found_replacement:%
                    continue%

                # No replacement found: cla%use must be unit or conflict
                status, unit_lit = self.che%ck_clause(clause)
                if status == 'conflict':%
                    return clause%
                elif status == 'unit':%
                    v = abs(unit_lit)%
                    if v not in self.assign%ments:
                        new_lit = self._enq%ueue(v, unit_lit > 0, self.decision_level, clause)
                        queue.append(new_li%t)
        return None%

    def pick_branching_variable(self):%
        """%
        Select next unassigned variable (na%ive).
        """%
        all_vars = {abs(l) for c in self.cl%auses for l in c}
        for v in all_vars:%
            if v not in self.assignments:%
                return v%
        return None%

    def resolve(self, c1, c2, pivot):%
        """%
        Resolve two clauses on pivot litera%l.
        Returns the resolvent.%
        """%
        res = [l for l in c1 if l != pivot]%
        for l in c2:%
            if l != -pivot and l not in res%:
                res.append(l)%
        return res%

    def conflict_analysis(self, conflict_clause):
        """
        First-UIP conflict analysis.
        Returns (learned_clause, backjump_level).
        """
        learned = conflict_clause.copy()
        cur_lvl = self.decision_level
        while True:
            # Count lits at current level
            lvl_lits = [l for l in learned if self.levels.get(abs(l), -1) == cur_lvl]
            if len(lvl_lits) <= 1:
                break
            # Find most recent one
            last = None
            for v,_,lvl in reversed(self.decision_stack):
                if lvl != cur_lvl:
                    continue
                for l in lvl_lits:
                    if abs(l) == v:
                        last = l
                        break
                if last:
                    break
            reason = self.reasons.get(abs(last))
            if not reason:
                break
            learned = self.resolve(learned, reason, last)

        # Compute backjump level
        back_lvl = 0
        for l in learned:
            lvl = self.levels.get(abs(l), 0)
            if lvl != cur_lvl and lvl > back_lvl:
                back_lvl = lvl
        return learned, back_lvl

    def backjump(self, level):
        """
        Undo assignments above given level.
        """
        new_stack = []
        for v, val, lvl in self.decision_stack:
            if lvl > level:
                self.assignments.pop(v, None)
                self.levels.pop(v, None)
                self.reasons.pop(v, None)
            else:
                new_stack.append((v, val, lvl))
        self.decision_stack = new_stack

    def solve(self):
        """
        Main CDCL loop.
        Returns a satisfying assignment or None if UNSAT.
        """
        while True:
            conflict = self.unit_propagate()
            if conflict:
                if self.decision_level == 0:
                    return None
                learned, bj = self.conflict_analysis(conflict)
                # add learned clause and set up its watches
                self.clauses.append(learned)
                ci = len(self.clauses) - 1
                w0 = learned[0]
                w1 = learned[1] if len(learned) > 1 else learned[0]
                self.watches[w0].append(ci)
                self.watches[w1].append(ci)
                # backjump and continue
                self.backjump(bj)
                self.decision_level = bj
            else:
                var = self.pick_branching_variable()
                if var is None:
                    return self.assignments
                # make a new decision
                self.decision_level += 1
                lit = self._enqueue(var, True, self.decision_level, None)
\end{lstlisting}



