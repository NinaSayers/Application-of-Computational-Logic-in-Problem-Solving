\chapter*{Introducción}\label{chapter:introduction}
\addcontentsline{toc}{chapter}{Introducción}
%Contexto histórico/social
El desarrollo de la lógica computacional como disciplina se enmarca en la revolución tecnológica del siglo XX, impulsada por la necesidad de resolver problemas complejos en ámbitos como la inteligencia artificial, la verificación de \textit{hardware} y \textit{software}, y la optimización industrial. La creciente demanda de sistemas automatizados capaces de procesar restricciones y tomar decisiones eficientes llevó a la comunidad científica a explorar métodos formales para modelar y resolver problemas combinatorios. En este escenario, la teoría de la complejidad computacional emergió como un pilar fundamental, especialmente tras la identificación de la clase NP-Completo por Cook en 1971, que transformó la comprensión de los límites de la computación.

%Antecedentes del problema científico
Los problemas con restricciones —aquellos que requieren satisfacer un conjunto de condiciones lógicas— han sido centrales en áreas como la planificación, la criptografía y el diseño de circuitos. El problema de satisfacibilidad booleana (SAT), demostrado por Cook como el primer problema NP-Completo, se convirtió en la piedra angular para estudiar la viabilidad de soluciones eficientes. Aunque los primeros algoritmos para SAT, como el método de Davis-Putnam (DP) y su evolución, Davis-Putnam-Logemann-Loveland (DPLL), sentaron las bases de los \textit{solvers}, su eficiencia se veía limitada por la explosión combinatoria en instancias complejas. La búsqueda y asignaci\'on secuencial de cláusulas unitarias, la falta de una estrategia de selecci\'on de variables y el retroceso (\textit{backtrack}) cronológico exponían claras debilidades, especialmente en problemas con miles de variables.

%Breve presentación de la problemática
A pesar de los avances, los SAT \textit{solvers} clásicos enfrentaban un desafío crítico: escalar sin sacrificar completitud. Esto motivó la búsqueda de mejoras heurísticas y estratégicas, como el aprendizaje de cláusulas y el \textit{backtrack} no cronológico, que culminaron en el surgimiento del paradigma \textit{Conflict-Driven Clause Learning} (CDCL). CDCL no solo optimizó la exploración del espacio de soluciones, sino que introdujo mecanismos para evitar repeticiones de conflictos, marcando un hito en la resolución práctica de problemas NP-Completos. 

%Sin embargo, la eficacia de estos algoritmos depende en gran medida de estrategias de selección de variables, como VSIDS (Variable State Independent Decaying Sum) y DLIS (Dynamic Largest Individual Sum), cuyas ventajas comparativas siguen siendo objeto de debate.

El núcleo de la eficiencia de los SAT \textit{solvers} modernos reside, sin lugar a dudas, en su capacidad para reducir el espacio de búsqueda de forma inteligente. Sin embargo, incluso con técnicas como CDCL, un desafío persiste: la selección óptima de variables. Esta elección determina la dirección en la que el algoritmo explora el árbol de decisiones, y una estrategia secuencial, o que no tenga en cuenta el posible impacto posterior a su asignaci\'on, puede llevar a ciclos de conflicto-reparación redundantes, incrementando exponencialmente el tiempo de ejecución. En problemas NP-Completos, donde el número de posibles asignaciones crece como $2^n$ (con $n$ variables), una heurística de selección inadecuada convierte instancias resolubles en minutos en problemas intratables.

%¿Por qué la selección de variables es crítica?
En CDCL, tras cada conflicto, el solucionador aprende una cláusula nueva para evitar repeticiones. No obstante, la eficacia de este aprendizaje depende de qué variables se eligieron para bifurcar el espacio de soluciones. Si se seleccionan variables irrelevantes o poco conectadas a los conflictos, las cláusulas aprendidas serán débiles o redundantes, limitando su utilidad. Así, la selección de variables no es solo una cuestión de orden, sino de calidad de la exploración.

Dos de las heurísticas de selección de variables son VSIDS (\textit{Variable State Independent Decaying Sum}) y DLIS (\textit{Dynamic Largest Individual Sum}). Ambas, son aproximaciones \textit{greedy}, dado que optimizan localmente (paso a paso) sin garantizar una solución global óptima. Su eficacia depende de cómo la estructura del problema se alinee con sus criterios. Por una parte, VSIDS asigna un puntaje a cada variable, incrementándolo cada vez que aparece en una cláusula involucrada en un conflicto. Periódicamente, estos puntajes se reducen (decaimiento exponencial), priorizando variables activas recientemente. Por otra parte, DLIS calcula, para cada literal (variable o su negación), el número de cláusulas no satisfechas donde aparece. Selecciona el literal con mayor frecuencia y asigna su variable correspondiente.

En los CDCL SAT \textit{solvers} se ha observado que el tiempo de ejecución puede seguir una distribución de ``cola pesada'' (\textit{heavy-tailed distribution}), lo que significa que el solucionador puede quedarse atascado en un camino de búsqueda improductivo por un tiempo prolongado. En aras de resolver este problema, surge la estrategia \textit{restart}, la cual borra parte del estado del \textit{solver} a intervalos determinados durante su ejecución. Su principal objetivo es reorientar la búsqueda y aprovechar el conocimiento acumulado mientras se evita profundizar en regiones improductivas del árbol de búsqueda. Al reiniciar, el solucionador puede escapar de una dirección de búsqueda desventajosa y tener una ``segunda oportunidad'' para encontrar una solución más rápidamente.



%VSIDS

%Fortaleza: Adaptabilidad dinámica. Al enfocarse en variables vinculadas a conflictos recientes, explota patrones locales en problemas estructurados (ej: verificaciones de circuitos).

%Debilidad: Puede ignorar variables críticas en regiones no exploradas del espacio.

%DLIS (Dynamic Largest Individual Sum):

%Mecanismo: Calcula, para cada literal (variable o su negación), el número de cláusulas no satisfechas donde aparece. Selecciona el literal con mayor frecuencia y asigna su variable correspondiente.

%Fortaleza: Basado en la estructura estática del problema, ideal para instancias con distribución uniforme de restricciones (ej: SAT aleatorio).

%Debilidad: No adapta su estrategia durante la ejecución, volviéndose ineficaz en problemas con dependencias ocultas o jerarquías.

%Actualidad
Hoy, aunque los SAT solucionadores basados en CDCL dominan aplicaciones críticas, desde la verificación formal de chips hasta la síntesis de programas, su rendimiento varía significativamente según el tipo de problema (p. ej., aleatorios vs. estructurados) y las heurísticas empleadas. Mientras VSIDS prioriza variables recientemente involucradas en conflictos —útil en problemas con alta estructura local—, DLIS enfatiza la frecuencia de aparición de literales, mostrando ventajas en dominios con distribución uniforme de restricciones. Esta dualidad plantea preguntas clave: ¿bajo qué métricas (tiempo de ejecución, memoria, escalabilidad) una estrategia supera a la otra? ¿Cómo influye la naturaleza del problema en su eficiencia?

%Novedad científica
Esta tesis aporta una comparación sistemática entre VSIDS y DLIS, alternando entre el uso de \textit{restart} dentro del entorno CDCL que ofrece el solucionador CaDiCaL, evaluando su desempeño en problemas heterogéneos (industriales, aleatorios y académicos). A diferencia de estudios previos, se integran métricas adaptativas que consideran no solo el tiempo de resolución, sino también el impacto de las caracter\'isticas de los problemas. Además, se propone un marco teórico amplio para comprender la evoluci\'on algor\'itmica de los CDCL SAT \textit{solvers}, entender algunas de las heur\'isticas que se emplean en los solucionadores modernos, espec\'ificamente en CaDiCaL, y clasificar problemas según su afinidad heurística, contribuyendo a la selección informada de algoritmos en aplicaciones reales.

%Importancia teórica y práctica
Teóricamente, este trabajo profundiza en la relación entre estructura de problemas y heurísticas, enriqueciendo la comprensión de CDCL. Prácticamente, ofrece directrices para ingenieros y desarrolladores de \textit{solvers}, optimizando recursos en áreas como la verificación de \textit{software} o la logística, donde minutos de mejora equivalen a ahorros millonarios.

%Diseño teórico
Como problema científico se plantea la ineficiencia de los SAT solucionadores ante problemas con distintas estructuras, asociada a la selección subóptima de variables, influenciada o no por t\'ecnicas de reinicio. El objeto de estudio se centrar\'a en algoritmos CDCL con estrategis VSIDS, DLIS y \textit{restart}. Esta tesis tiene como objetivos:
\begin{itemize}
    \item Analizar el impacto de VSIDS y DLIS, con y sin reinicio, en el rendimiento de CDCL.
    \item Establecer correlaciones entre tipos de problemas y heurísticas.
    \item Establecer correlaiones entre tipos de problemas y resultados de las heur\'isticas.
\end{itemize}
El campo de acci\'on de esta tesis versa sobre la Lógica computacional aplicada a la resolución de problemas con restricciones.
Como hipótesis se plantea que: El rendimiento de VSIDS y DLIS con y sin \textit{restart} varía significativamente según la densidad de restricciones, el tama\~no promedio de cl\'ausula y la cantidad de variables.
Esta investigación busca no solo esclarecer el debate entre VSIDS y DLIS, sino también sentar bases para el diseño de heurísticas adaptativas, impulsando la próxima generación de resolvedores.

%Estructuración del trabajo
El documento se organiza en cinco capítulos:
\begin{itemize}
    \item \textbf{Cap\'itulo 1} Revisión teórica de los principales algoritmos usados en los solucionadores SAT, algunas heur\'isticas empleadas en los \textit{solvers} modernos haciendo \'enfasis en CaDiCaL, y de las categor\'ias de problemas usadas en los experimentos que se llevaron a cabo.
    \item \textbf{Cap\'itulo 2} Detalles de implementaci\'on de las heur\'isticas en CaDiCaL, del generador de problemas y de los an\'alisis estad\'isticos empleados. Explicaci\'on detallada de los algoritmos empleados y exposici\'on de bibliotecas usadas.
    \item \textbf{Cap\'itulo 3} Resultados de los experimentos.
    \item \textbf{Cap\'itulo 4} Conclusiones y recomendaciones.
\end{itemize}

